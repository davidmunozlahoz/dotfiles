Vim�UnDo��	�#��RܧFL�1~�B9�fۀ�/�g��5c��_�1����1�Vc���01^
\maketitle;\section{Obtaining projection bands in $\reg E$ from a banddecomposition in $E$}@Let $E$ be a Dedekind complete Banach lattice. Then the space ofDregular operators $\reg E$ is also a Banach lattice. Suppose that weChave a family of bands in $E$, denoted by $\{F_i\}_{i \in I}$ for a certain index set $I$, such that\[E=\bigoplus_{i \in I} F_i,\]7meaning that any $x \in E$ can be expressed uniquely as\[2x=\sum_{i \in I}^{} x_i \text{ where } x_i \in F_i\]Aand only countably many of the $x_i$ are not zero. Since $E$ is a>Dedekind complete Banach lattice, there exist band projections@$P_i:E\to E$ onto $F_i$ for $i \in I$. Then any $x \in E$ can be
written as\[x=\sum_{i \in I}^{} P_i x\]Bwhere again only countably many terms will be different from zero.CNow given $T \in \reg E$, in general we can express (if $id:E\to E$denotes the identity map)"\begin{equation}\label{eq:fulldec}>T=id T id=\left( \sum_{i \in I}^{} P_i \right) T\left( \sum_{j4\in I}^{} P_j \right) =\sum_{i,j \in I}^{} P_i T P_j\end{equation}Jwhere this equality makes sense pointwise, meaning that for each $x \in E$?the previous expression evaluated at $x$ yields a countable andGconvergent series. We want to prove that choosing some pairs of indicesA$(i,j)$ in the previous sum yields a band projection on $\reg E$.\begin{lem}F    Let $A \subseteq I \times I$ be a family of pairs of indices, then    the map    \[    \begin{array}{cccc}7    \mathcal{P}: & \reg E & \longrightarrow & \reg E \\=        & T & \longmapsto & \sum_{(i,j) \in A}^{} P_i TP_j \\    \end{array}    \]    is a band projection.	\end{lem}\begin{obs}W    Before proceeding to the proof, we need to introduce some notation to deal with the    indices. Define    \[    \begin{array}{cccc}0    \pi _1: & I\times I & \longrightarrow & I \\$        & (i,j) & \longmapsto & i \\$    \end{array}\quad \text{and}\quad    \begin{array}{cccc}0    \pi _2: & I\times I & \longrightarrow & I \\$        & (i,j) & \longmapsto & j \\    \end{array}.    \]A    We will denote $I_0=\pi _1(A)$, $J_0=\pi _2(A)$ and also $A_i    =\pi_2(\piM    _1^{-1}(i)\cap A)$ for any $i \in I$ and $A^{j}=\pi _1(\pi_2 ^{-1}(j)\cap    A)$ for any $j \in I$.	\end{obs}
\begin{proof}<    First of all, we need to clarify what the expression forE    $\mathcal{P}(T)$ actually means. We should interpret it as we didF    in \eqref{eq:fulldec}, that is, for $x \in E$, $\mathcal{P}(T)(x)$?    defines an element of $E$, because the sum is countable andE    convergent (it converges absolutely because we are removing terms2    from $Tx$ as expressed in \eqref{eq:fulldec}).1    So we have a map $\mathcal{P}(T):E\to E$. For;    $\mathcal{P}$ to be well-defined, we have to check that@    $\mathcal{P}(T)$ is linear and regular. Note that, using the7    notation introduced for the indices, we can express    \[?    \mathcal{P}(T)x=\sum_{i \in I_0}^{} P_i T\bigg( \sum_{j \in        A_i}^{} P_jx\bigg)    \]H    which is clearly linear. Moreover, for any $T\ge 0$ and $x \in E_+$:    \[?    \mathcal{P}(T)x=\sum_{i \in I_0}^{} P_i T\bigg( \sum_{j \in<        A_i}^{} P_jx\bigg)\le \sum_{i \in I_0} P_i(Tx)\le Tx    \]A    so $\mathcal{P}(T)\le T$ and thus $\mathcal{P}(T)$ is regularC    (moreover, if $I:\reg E\to \reg E$ is the identity map, we have@    proved that $\mathcal{P}\le I$, which will be useful later).    B    We have that $\mathcal{P}$ is a well-defined map. Next we haveC    to prove that it is a projection. That is linear is clear if weC    think how $\mathcal{P}(\lambda T+\mu S)$ acts pointwise on each@    element. To see that it is a projection, note that given any"    subset $J\subseteq I$, we have    \[B        \left( \sum_{k \in J} P_k \right) \left( \sum_{j \in J}^{}>        P_jx \right) =\sum_{k \in J}^{} P_k \left( \sum_{j \in?        J}^{} P_jx \right) =\sum_{k \in J}^{} \sum_{j \in J}^{}&        P_kP_jx=\sum_{j \in J}^{} P_jx    \]E    where we have used the linearity and continuity of the operators.
    Therefore    \begin{align*}<        \mathcal{P}(\mathcal{P}(T))&=\sum_{i \in I_0}^{} P_iA    \mathcal{P}(T)\sum_{j \in A_i}^{} P_j=\sum_{i \in I_0}^{} P_iC    \left( \sum_{k \in I_0}^{} P_k T\sum_{l \in A_k}^{} P_l \right):        \sum_{j \in A_i}^{} P_j\\&=\sum_{i \in I_0}^{} P_iE     T\sum_{l \in A_i}^{} P_l \sum_{j \in A_i}^{} P_j=\mathcal{P}(T).    \end{align*}F    Finally, to see that $\mathcal{P}$ is a band projection we have to=    check that $0\le \mathcal{P}\le I$. We already proved the    inequality:    $\mathcal{P}\le I$. If $T\ge 0$ and $x \in E_+$, then:    \[<    \mathcal{P}(T)x=\sum_{i \in I_0}^{} P_i T \left( \sum_{j    \in A_i}^{} P_j x \right) ,    \]E    where each $\sum_{j \in A_i}^{} P_j x $ is positive (because bandF    projections are positive and the possibly infinite sum of positiveF    elements is positive, since the lattice operations are continuous)9    so $T\sum_{j \in A_i}^{} P_j x $ is positive and thusB    $\mathcal{P}(T)x\ge 0$. This proves that $\mathcal{P}(T)\ge 0$E    and, since this is true for any $T\ge 0$, we have $\mathcal{P}\ge    0$.\end{proof}@So for each $A \subseteq I \times I$ we have the band projection\[    \begin{array}{cccc}9    \mathcal{P}_A: & \reg E & \longrightarrow & \reg E \\=        & T & \longmapsto & \sum_{(i,j) \in A}^{} P_i TP_j \\    \end{array}.\]$Let us denote the associated band byF$\mathcal{B}_A=\text{range}(\mathcal{P}_A)$. We can give the followingdescription of this band.\begin{lem}&    Let $A \subseteq I \times I$, then    \[=    \mathcal{B}_A=\bigg\{\, S \in \reg E \mid S(F_j)\subseteqE    \bigoplus_{i \in A^{j}}F_i \text{ for every } j \in I\, \bigg\} ,    \]=    where we define $\sum_{i \in A^{j}}^{} F_i$ to be $\{0\}$1    when $A^{j}=\O$ (i.e., when $j \not\in J_0$).	\end{lem}
\begin{proof}B    Let $T \in \reg E$. If $x \in F_j$ with $j \not\in J_0$, it is(    clear that $\mathcal{P}(T)(x)=0$, soD    $\mathcal{P}(T)(F_j)=0=\bigoplus_{i \in A^{j}}F_j$, according to0    our convention. If instead $j \in J_0$, then    \[<    \mathcal{P}(T)(x)=\sum_{i \in I_0}^{} P_iT\left( \sum_{j?    \in A_i}^{} P_j(x) \right) =\sum_{i \in A^{j}}^{} P_iTx \in     \bigoplus_{i \in A^{j}} F_i,    \]G    and again $\mathcal{P}(T)(F_j) \subseteq \bigoplus_{i \in A^{j}}^{}B    F_i$. This proves that any $T \in \mathcal{B}_A$ satisfies the    required condition.8    Conversely, suppose that $S \in \reg E$ is such thatC    $S(F_j)\subseteq \bigoplus_{i \in A^{j}}A_i$ for any $j \in I$.    Then for any $x \in E$:    \[?    \mathcal{P}(S)(x)=\sum_{(i,j)\in A}^{} P_iSP_jx=\sum_{j \in@    J_0}^{} \left(\sum_{i \in A^{j}}^{} P_i\right) SP_jx=\sum_{j    \in J_0}^{} SP_jx    \]8    because $P_jx \in F_j$, and by hypothesis $SP_jx \in@    \sum_{i \in A^{j}}^{} F_i$. Now using that $SP_jx=0$ when $j    \not\in J_0$ we get    \[?    \mathcal{P}(S)(x)=\sum_{j \in J_0}^{} SP_jx=S\left( \sum_{jF    \in J_0}^{} P_j x \right) =S\left( \sum_{j \in I}^{} P_j x \right)
    =S(x).    \]/    Hence $S=\mathcal{P}(S) \in \mathcal{B}_A$.\end{proof}\begin{ex}[Two bands]E    Let us consider the simplest case, in which we have a single bandD    $F\subseteq E$, which means that we can decompose the space intoB    $E=F\oplus F^{d}$, where $F^{d}$ is the disjoint complement ofF    $F$. Let $P:E\to E$ be the projection onto $F$ and let $Q=id-P$ be?    the projection onto $F^{d}$. Then any $T \in \reg E$ can be    written as    \[    T=PTP+PTQ+QTP+QTQ.    \]D    Let us also associate the indices $P_1=P$ and $P_2=Q$. According@    to the previous lemmas, for each possible subset $A\subseteqC    \{(1,1),(1,2),(2,1),(2,2)\}$ we have a band projection in $\regA    E$. Discarding the trivial cases of the zero and the identityE    (corresponding to the empty and total set, respectively), we have#    the following 14 possibilities:        \begin{center}         \begin{tabular}{|c|c|c|}            \hlineC            $A$ & $\mathcal{P}_A(T)$ & $S \in \mathcal{B}_A$ if and$            only if \\ \hline \hlineE            $(1,1)$ & $PTP$ & $S(F)\subseteq F, S(F^{d})=0$ \\ \hlineI            $(2,1)$ & $QTP$ & $S(F)\subseteq F^{d}, S(F^{d})=0$ \\ \hlineE            $(1,2)$ & $PTQ$ & $S(F^{d})\subseteq F, S(F)=0$ \\ \hlineI            $(2,2)$ & $QTQ$ & $S(F^{d})\subseteq F^{d}, S(F)=0$ \\ \hline            \hline6            $(1,1),(2,2)$ & $PTP+QTQ$ & $S(F)\subseteq/            F,S(F^{d})\subseteq F^{d}$\\ \hlineA            $(1,1),(2,1)$ & $PTP+QTP=TP$ & $S(F^{d})=0$ \\ \hlineF            $(1,1),(1,2)$ & $PTP+PTQ=PT$ & $S(E)\subseteq F$ \\ \hline6            $(2,1),(1,2)$ & $QTP+PTQ$ & $S(F)\subseteq0            F^{d},S(F^{d})\subseteq F$ \\ \hlineJ            $(2,1),(2,2)$ & $QTP+QTQ=QT$ & $S(E)\subseteq F^{d}$ \\ \hline=            $(1,2),(2,2)$ & $PTQ+QTQ=TQ$ & $S(F)=0$ \\ \hline            \hline7            $(1,1),(1,2),(2,1)$ & $PTP+PTQ+QTP=T-QTQ$ &+            $S(F^{d})\subseteq F$ \\ \hlineF            $(1,1),(1,2),(2,2)$ & $PTP+PTQ+QTQ=T-QTP$ & $S(F)\subseteq            F$ \\ \hline7            $(1,1),(2,1),(2,2)$ & $PTP+QTP+QTQ=T-PTQ$ &/            $S(F^{d})\subseteq F^{d}$ \\ \hlineF            $(1,2),(2,1),(2,2)$ & $PTQ+QTP+QTQ=T-PTP$ & $S(F)\subseteq            F^{d}$ \\ \hline        \end{tabular}            \end{center}        \end{ex}#\begin{ex}[Infinite bands: $\el 1$]>    Consider the space $\el 1$, and denote by $e_n$ the $n$-thE    coordinate vector. Then we have the following band decomposition:    \[6    \el 1=\bigoplus_{n=1}^{\infty } \left< e_n \right>    \]E    where the projection associated to each band is $P_n:\el 1\to \el2    1$, the projection onto the $n$-th coordinate.E    In $\el 1$ every bounded operator is regular, so $\reg{\el 1}$ isA    just the set of bounded operators $L(\el 1)$. It is useful toD    think of bounded operators $T \in L(\el 1)$ as infinite matrices'    $(a_{ij})_{i,j=1}^{\infty }$, whereC    $Te_j=\sum_{i=1}^{\infty}a_{ij}e_i$ for $j \in \N$. Then $T \in2    L(\el 1)$ if and only if the associated matrix)    $(a_{ij})_{i,j=1}^{\infty }$ verifies    \[:    \sup_{j \in \N} \sum_{i=1}^{\infty} |a_{ij}|< \infty .    \]A    The matrices associated with the projections are $P_n=(\deltaE    _{in}\delta _{jn})_{i,j=1}^{\infty }$, that is, they are entirely=    zero except for the entry $(n,n)$, where they have a $1$.C    Let $A \subseteq \N \times \N$ be a family of indices. Then the    associated band projection    \[    \begin{array}{cccc}=    \mathcal{P}_A: & L(\el 1) & \longrightarrow & L(\el 1) \\=        & T & \longmapsto & \sum_{(m,n) \in A}^{} P_m TP_n \\    \end{array},    \]5    takes the matrix $(a_{ij})$ of $T$ and only keepsD    the entries selected in the pairs of indices of $A$, setting all    the others to 0.7    But we can say more about the bands of $\el 1$. Let?    $\mathcal{P}:\reg{\el 1}\to \reg{\el 1}$ be \emph{any} band=    projection. By linearity and continuity, $\mathcal{P}$ isF    completely determined by the image of the matrices $E_{kl}=(\deltaD    _{ik} \delta _{jl})_{i,j=1}^{\infty }$ (that is, $E_{kl}$ is theD    matrix with zeros everywhere except for the entry $(k,l)$, where    it has a one).H    { \color{blue}\textbf{Aclaración} ¿Es este último hecho evidente?E        ¿Alguna referencia? Parece una comprobación directa, aunque+    algo pesada, y es bastante intuitivo. }C    Since $\mathcal{P}$ is a band projection, and $E_{kl}\ge 0$, weD    have $0\le \mathcal{P}(E_{kl})\le E_{kl}$, which only leaves theD    possibility $\mathcal{P}(E_{kl})=\alpha _{kl}E_{kl}$, with $0\le#    \alpha _{kl}\le 1$. But $\alpha@    _{kl}E_{kl}=\mathcal{P}(E_{kl})=\mathcal{P}^2(E_{kl})=\alphaC    _{kl}^2 E_{kl}$, so $\alpha _{kl}$ is either 0 or 1. Therefore,D    $\mathcal{P}$ puts zeros or ones in the entries of the matrix ofF    $T$ or, what is the same, $\mathcal{P}=\mathcal{P}_A$ where $A$ is9    formed by the pairs $(k,l) \in \N\times \N$ such that!    $\mathcal{P}(E_{kl})=E_{kl}$.C    {\color{blue} \textbf{Pregunta abierta} Parece que esto últimoF        tiene que ser cierto en cualquier retículo de Banach con base     1-incondicional. Comprobar.}\end{ex}7\subsection{Some algebraic relations between the bands}BWe want to characterize some algebraic relations between the bands@$\mathcal{B}_A$ in terms of the index sets $A$ that define them.\begin{lem}}    Let $A,B \subseteq I\times I$ be two families of indices. If $B\subseteq A$, then $\mathcal{B}_B\subseteq \mathcal{B}_A$.	\end{lem}
\begin{proof}D    If $B\subseteq A$ then $B^{j}\subseteq A^{j}$ for any $j \in I$,    which implies    \[E    \bigoplus_{i \in B^{j}} F_i \subseteq \bigoplus_{i \in A^{j}}F_i.    \]6    Thus, if $T \in \mathcal{B}_B$, for any $j \in I$:    \[T    T(F_j)\subseteq \bigoplus_{i \in B^{j}} F_i \subseteq \bigoplus_{i \in A^{j}}F_i    \]-    which implies that $T \in \mathcal{B}_A$.\end{proof}ALet $T,S \in \reg E$ be two regular operators. We say that $T$ isF\emph{left orthogonal} to $S$, and that $S$ is \emph{right orthogonal}Fto $T$, if $TS=0$. We denote this as $T\aperp S$ (where the `a' standsAfor an algebraic notion of orthogonality, not to be confused withF$T\perp S$ used to denote that $T$ and $S$ are disjoint in the lattice=sense). Given two families of regular operators $\mathcal{F},;\mathcal{G} \subseteq \reg E$, we denote $\mathcal{F}\aperpF\mathcal{G}$ if for any $T \in \mathcal{F}$ and $S \in \mathcal{G}$ wehave $T\aperp S$.\begin{lem}E    Let $A,B \subseteq I\times I$ be two families of indices. If $\pi7    _2(A)\cap \pi _1(B)=\O$, then $\mathcal{B}_A \aperp    \mathcal{B}_B$.	\end{lem}
\begin{proof}E    The fact that $\pi _2(A)\cap \pi _1(B)=\O$ implies that, given $j?    \in I$, for any $i \in B^{j}$ we have $A^{i}=\O$. Therefore    \[B    S(T(F_j))\subseteq S\left( \bigoplus_{i \in B^{j}} F_i \right)F    =\bigoplus_{i \in B^{j}} S(F_i) \subseteq  \bigoplus_{i \in B^{j}}#    \bigoplus_{k \in A^{i}} F_k =0,    \]9    for any $j \in I$, which readily implies $S(T(E))=0$.\end{proof}E{\color{blue} \textbf{Pregunta abierta} Parece que los recíprocos deElos lemas anteriores son ciertos, si tenemos alguna propiedad que nosGpermita encontrar operadores que envían una banda a dentro de otra (noBhace falta ni que sea de manera inyectiva o sobreyectiva). Esto es7definitivamente cierto en $\el 1$. ¿Lo es en general?}C% \section{If $E$ has a basis, $\mathcal{B}_A$ are all the possible% bands of $\reg E$}5�_�0����11Vc���052�0215��